\section{Introduction}
\label{sec:Introduction}
Cognitive computational trust modelling started from the classical introduction from Castlefranchi and Falcone \cite{Castelfranchi1998}, which established the initial view of representing trust as a collection of beliefs and intentions about delegating a certain task to the trustee. It also provided the vision that trust is an emergent concept mainly composed by our belief in the trustor's ability for the task in question, and it's willingness to perform the task in question. In recent years some attempts have been made to improve upon this aspects\cite{Piunti2012, Sutcliffe2012, Singh2011,Castelfranchi2010, Pinyol2009, Sabater2006, Neville2004}, further addressing issues like reputation\cite{Sabater2006}, and belief representation \cite{Pinyol2009, Piunti2012}.

For the purpose of this thesis, we sought out to create a model that showcased the multidimensionality of trust and how much do the different features of trust affect a requested trust evaluation. To do this we based our work on the Repage \cite{Sabater2006} architecture, but we simplified the memory component, relying more on a rule-based approach than a logic-based one, with stricter restrictions. In the following paragraphs we will describe (insert bla bla)