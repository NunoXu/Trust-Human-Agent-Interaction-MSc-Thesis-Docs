\section{Background}
\label{sec:Background}

% Reference 3 types of agents that may be applicable to the project: 
% - Social Agents
% - Afective Agents
% - Anthromorphic Agents
% --- Embodied Conversational Agents (ECAs)

% When talking about trust and reputation mention the differences between both.
% Tell what is a trust model and what is a reputation model

Before discussing related work and our solution to the problem, we will present the main concepts concerning Trust.

\subsection{Trust}
\label{subsec:Trust}
Trust is regarded throughout the literature as one of the fundamental components of human society, being essential in cooperative and collaborative behaviour. So it has been studied in a multitude of disciplines, from Psychology and Sociology, to Philosophy and Economy\cite{Rousseau1998, Jones1997, Sabater2005}. For that reason, it is no wonder that it acquired a very large number of different definitions throughout the years of study, causing the problem of not existing a consensus on a definition of trust\cite{Castelfranchi2010}. But in the scope of this project, the most relevant basis is the dyadic definition of trust, where it is defined as: 'an orientation of an actor (the truster) toward a specific person (the trustee) with whom the actor is in some way interdependent' (taken from \cite{Simpson2007}), as we want to focus on interpersonal relationships. This definition has been expanded throughout the literature in Computational Trust, often adapted to fit the context or scope of the work, but from them three main definitions were spawned:
\begin{itemize}
	\item First Gambetta\cite{Gambetta1988} defined trust as follows: 'Trust is the \textit{subjective probability} by which an individual A, \textit{expects} that another individual, B, performs a given action on which its \textit{welfare depends}' (taken from \cite{Castelfranchi2010}). This is accepted by most authors as one of the most classical definitions of trust, but it is restrictive in the sense that it is uni-dimensional, as it only refers to predictability of the trustor, and does not take into account competence.
	
	\item Then Marsh\cite{Marsh1994} was the first author to formalize trust as a measurable Computational Concept, continuing the perspective of reducing trust to a numerical value set by Gambetta \cite{Gambetta1988}, but also adding that: X trusts Y if, and only if, 'X \textit{expects} that Y will behave according to X's best interest, and will not attempt to harm X' (taken from \cite{Castelfranchi2010}).
	
	\item Castelfranchi and Falcone then defined a new definition and paradigm for Computational Trust, introducing a Cognitive aspect to it\cite{Castelfranchi1998}. They define Trust as the mental state of the trustor and the action in which the trustor refers upon the trustee to perform. This is the definition of trust that we will adopt throughout the rest of the report.
\end{itemize}

\subsubsection{Castelfranchi and Falcone's Trust}
\label{subsubsec:CastelfranchiTrust}
More explicitly, Castelfranchi and Falcone\cite{Castelfranchi1998} state that Trust can be defined with a central core, composed by a five-part relation, between the trustor and trustee, the context where they are inserted in, the action and outcome done by the trustee, and the goal of the trustor. This defines Trust as goal-oriented, contextual, and multi-dimensional, as from the point of view of the trustor, it varies not only on the trustee, but also from the overall context, the action the trustee is being delegated for, and the particular goal of the trustor. For example, if the goal of the trustor is simple to perform and not very critical to him, he may be more willing to delegate the task, and trust another agent to perform such task. It's important to note, that Trust must imply that the trustor is taking some kind of risk by delegating a task to the trustee. Be it because the trustee may not able to perform the task, or that he will purposely ruin the task and go against your goals.


\subsection{Reputation}
\label{subsec:Reputation}
Reputation is also a concept that appears very often linked with trust in the literature, specially since most models created for representing trust have been focused on \acp{MAS}, where trust is influenced not only by the image one has of the subject, but also by what other agents say about it.

For the purpose of this report, reputation of an agent is defined as the combined trust opinion that the other agents that have manifested or provided about a particular subject. This reputation can and should be different from the individual image that various agents may have about the subject.
