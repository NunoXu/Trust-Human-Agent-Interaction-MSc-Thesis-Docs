\section{Background}
\label{sec:Background}

% Reference 3 types of agents that may be applicable to the project: 
% - Social Agents
% - Afective Agents
% - Anthromorphic Agents
% --- Embodied Conversational Agents (ECAs)

% When talking about trust and reputation mention the differences between both.
% Tell what is a trust model and what is a reputation model

Before discussing related work and our solution to the problem, we will present the main concepts concerning Trust.

\subsection{Trust}
\label{subsec:Trust}
Trust is regarded throughout the literature as one of the fundamental components of human society, being essential in cooperative and collaborative behaviour. So it has been studied in a multitude of disciplines, from Psychology and Sociology, to Philosophy and Economy\cite{Jones1997, Sabater2005}. For that reason, it is no wonder that it acquired a very large number of different definitions through the years of study, causing the problem of not existing a consensus on a definition of trust\cite{Castelfranchi2010}. In the field relevant to us, Computational Trust, three main definitions exist:
\begin{itemize}
	\item 'Trust is the \textit{subjective probability} by which an individual A, \textit{expects} that another individual, B, performs a given action on which its \textit{welfare depends}' (taken from \cite{Castelfranchi2010}), by Gambetta \cite{Gambetta2015}. This is accepted by most authors as one of the most classical definitions of trust.
	
	\item Marsh was the first author to formalize trust as a Computational Concept\cite{Marsh1994}, continuing the perspective of reducing trust to a numerical value, but also adding that: X trust Y if and only if 'X \textit{expects} that Y will behave according to X's best interest, and will not attempt to harm X' (taken from \cite{Castelfranchi2010}).
	
	\item Castelfranchi and Falcone then defined a new definition and paradigm for Computational Trust, introducing a Cognitive aspect to it\cite{Castelfranchi1998}. They define Trust as the mental state of the trustor and the action in which the trustor refers upon the trustee to perform. This is the definition of trust that we will adopt throughout the rest of the report.
\end{itemize}

\subsubsection{Castelfranchi and Falcone's Trust}
\label{subsubsec:CastelfranchiTrust}
More explicitly, Castelfranchi and Falcone 


\subsection{Reputation}
\label{subsec:Reputation}
Reputation is also a concept that appears very often linked with trust in the literature, specially since most models created for representing trust have been focused on \acp{MAS}, where trust is influenced not only by the image one has of the subject, but also by what the community says about it.

For the purpose of this report, reputation is defined as the trust opinion that the shared voice of the community of agents has about a particular subject. 
