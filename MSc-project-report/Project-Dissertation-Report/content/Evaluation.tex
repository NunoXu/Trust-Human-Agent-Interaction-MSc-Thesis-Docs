\section{Evaluation}
\label{sec:Evaluation}
As stated in Section \ref{sec:Introduction}, the project's evaluation will be done by integrating the developed modules in a agent, now finishing development, that is capable of acting as a player in the \textit{Split or Steal} scenario, introduced in the British television show \textit{Golden Balls}\cite{Wikipedia.Golden.Balls}. The scenario involves two players and stands as follows:
\begin{itemize}
	\item A large sum of money is prized for the game;
	\item Each player receives two balls, one has `Steal' written inside while the other has `Split';
	\item The balls can be stealthily open by the players, so only each player knows is written in what ball;
	\item The players then have some time to discuss and negotiate between one another;
	\item Finally the players choose one of their balls and show their content simultaneously and the game finalizes, giving one of the following results:
	\begin{itemize}
		\item If both players choose `Split', they split the prize money evenly;
		\item If one picks `Steal' and the other `Split', the stealer gets all the prize money;
		\item If both pick `Steal', they both lose all the prize money.
	\end{itemize}	
\end{itemize}

From a game theory standpoint, this scenario is a variation on \textit{Prisoner's Dilemma}, described in Section \ref{subsubsec:PrisonersDilemma}, where defecting only weakly dominates cooperating, because if an opposing player picks `Steal' we get nothing whether we pick `Steal' or `Split', so `Steal' only dominates if the opposing player picks `Split'.

The negotiation phase of the game is the one we are going to focus on our evaluation procedures. 