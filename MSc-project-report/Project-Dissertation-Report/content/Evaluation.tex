\section{Evaluation}
\label{sec:Evaluation}

We developed an initial evaluation plan for the thesis, but as the project is designed to be able perform in various conflict and collaboration scenarios, we hope to improve upon 

\subsection{Evaluation Plan}
As stated in Section \ref{sec:Introduction}, the project's evaluation will be done by integrating the developed modules in a agent, now finishing development, that is capable of acting as a player in the \textit{Split or Steal} scenario, introduced in the British television show \textit{Golden Balls}\cite{Wikipedia.Golden.Balls}. The scenario involves two players and stands as follows:
\begin{enumerate}
	\item A large sum of money is prized for the game;
	\item Each player receives two balls, one has `Steal' written inside while the other has `Split';
	\item The balls can be stealthily open by the players, so only each player knows is written in what ball;
	\item \label{enum:split.steal.neg}The players then have some time to discuss and negotiate between one another;
	\item Finally the players choose one of their balls and show their content simultaneously and the game finalizes, giving one of the following results:
	\begin{itemize}
		\item If both players choose `Split', they split the prize money evenly;
		\item If one picks `Steal' and the other `Split', the stealer gets all the prize money;
		\item If both pick `Steal', they both lose all the prize money.
	\end{itemize}	
\end{enumerate}

From a game theory standpoint, this scenario is a variation on \textit{Prisoner's Dilemma}, described in Section \ref{subsubsec:PrisonersDilemma}, where defecting only weakly dominates cooperating, because if an opposing player picks `Steal' we get nothing whether we pick `Steal' or `Split', so `Steal' only dominates if the opposing player picks `Split'. We will focus evaluation in Step \ref{enum:split.steal.neg} of the \textit{Split or Steal} scenario, the negotiation phase, where there is most probable for the agent and user to perform some spoken interaction.

Evaluation will be performed through individual user testing of the agent, and we will try to gather at least 30 participants to ensure statistical viability. The testers will then be separated into two equally distributed groups, which we will designate by \textit{Group A} and \textit{Group B}. \textit{Group A} will be the control group.  The following steps will be performed for each user:

\begin{enumerate}
	\item Perform a series of \textit{Split or Steal} games with an individual user and the agent as players; in \textit{Group A} the agent will play \textbf{without} our modules and in \textit{Group B} with them.
	\item After interaction with the agent, the user will fill out a questionnaire to assert the value of Trust has in the Agent, in a range from 0 to 100; the questionnaire to be used will be the described in Section \ref{subsec:Related work:The Perception and Measurement of Human-Robot Trust}; whether we will use the complete version or the one described in %TODO insert ref to architecture
	is still to be decided.
\end{enumerate}

We will then compare the averaged Trust value of both groups, and if the value of \textit{Group B} is greater by a significant margin, it will provides positive feedback to the decision making module. Additionally we will check how closely did the questionnaire answers matched with the model created in the agent by the user trust module % TODO fix this, give better name
, providing a measure of accuracy of the model created.


We hope to improve the scenario to represent tasks where collaboration 