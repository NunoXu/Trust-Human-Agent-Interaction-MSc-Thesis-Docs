\section{Introduction}
\label{sec:Introduction}
Trust is \ac{HAI}.
% TODO - Insert motivation on the topic:
% - while a lot of reasearch has been done in modeling trust and in automation, little  research trust in HAI has been found
% - trust is an important component of the social fabric, and essential if we want to have collaboration
% - mention that people tend to "anthropomorphize" computers, 





\subsection{Objectives}
\label{subsec:Objectives}
% we aim to create a generic agent model that combines the already extensive reasearch in trust modeling in MAS systems with a decision making module with strategies to improve trust in the agent.
% we want to focus on human to agent interaction. so improve and model human trust.
% find out which parameters and features can be explicitly chosen to human modeling.
This project's purpose will be to:
\begin{itemize}
	\item Create a trust model capable of representing human trust towards the agent;
	\item Develop a decision making module that aims to suggest actions that maximize trust towards the agent;
	\item 
\end{itemize}





In section X we introduce the background research to understand the concepts of rapport, social robotics and machine learning. In section Y, it will be described the state-of-art rapports agents developed by X, Y and Z. The proposed solution and how its quality will be evaluated will be described in section W. Lastly, the final conclusions and the main concerns will be address in section U.

The remainder of this paper is organized as follows. In
the next section, we discuss related work on learning social behavior synthesis models. We introduce the IPL approach in Section III. In Sections IV and V, we describe, respectively, the setup and the results of an experiment on the synthesis of backchannel timings. We conclude with Section VI.
