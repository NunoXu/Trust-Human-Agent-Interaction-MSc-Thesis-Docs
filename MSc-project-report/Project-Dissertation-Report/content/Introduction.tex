\section{Introduction}
\label{sec:Introduction}

% TODO Rapport should be defined here?
% IS it cultural?

% --Rapport and facial expressions
When you engage in a good conversation with someone, that feeling of flow and connection is what formally known as rapport.
% ---

% Robot behaviour toolkit
These stud- ies demonstrate the necessity of robots displaying appropriate so- cial behaviors to effectively achieve social, cognitive, and task out- comes in human-robot interaction
%

Robots are increasingly becoming part of our society and its presence has been proven to impact our lives. But does any of us remember a remarkable interaction with a robot as we are able to recall one with a person? What makes one conversation memorable? How can we design robots that can achieve something that has so much impact and yet, people do so easily?

In order to ask these questions, the \ac{HRI} research community has been exploring agents capable of responding emotionally and more humanly in dyadic interactions to create empathy and positivity. Formally, this phenomenon is known in psychology as building rapport which will be more detailed in Section \ref{subsec:Background}. 

For that purpose, several studies were conducted to identify how people can be manipulated using different verbal and non-verbal strategies. There is evidence in these studies that rapport agents can make people feel: more connected \cite{Rosenthal-vonderPutten2013}, less tension ~\cite{Wang2010}, less embarrassed \cite{Kang2009}, mutual trustworthiness \cite{Kang2009}. 

Rapport agents can be applied in social robots in several domains such as negotiation \cite{Nadler2003}, child care \cite{Burns1984}, therapeutic sessions \cite{Lisetti2013}, family companions \cite{Andrist2014, Bickmore2005}, hospitals \cite{Bull1981} \cite{Kang2005}, weight loss \cite{Burroughs2007}, children with autism \cite{Feil-Seifer2009, Scassellati2012}, companions at home \cite{Fasola2012, Marti2006} and several other examples.

But, despite present efforts there are still important issues to be addressed, namely:
\begin{itemize}
	\item Lack of coordination during interactions due to poor timing predictability;
	\item Most of the work in the area is focused on short-term rapport and therefore does not take into account how relationships evolve and how rapport strategies are molded by it;
	\item Lack of rapport datasets;
	\item There is not a common framework nor model to manage rapport;
\end{itemize}

Lastly, we will  develop a rapport component using SERA frameworkd (XX) and EMotive headY System (EMYS) and demonstrate our results in a negotiation game scenario how the developed agent can affect the emotional state and affect the trust level of the opponent. The chosen game is Split Or Steal during the negotiation phase scenarios and in trust such as Split Or Steal (YY).

% falar das maiores dificuldades com maior detalhe

%
% An instant rapport approach is useful for building systems that are initially attractive to users; but a system that signals increasing familiarity and intimacy through its linguistic and nonverbal behaviors may encourage users to stay with the system over a longer period of time.  [Rapport and facial expresions]
%


%(e.g. facial expression mirroring \cite{Dimberg2000, Hess1999})
%powerful effect nego- tiation [15], counseling [19] and education [4]. TowardsDyadicComputationalModel

\subsection{Objectives}
\label{subsec:Objectives}

The main goals of the proposed project are:
\begin{itemize}
	\item Create a rapport model to be integrated with the SERA unified framework;
	\item Create a rapport model based on machine learning algorithms;
	\item Create a rapport model using Wizard of Oz
	
	
	Integrate a rapport agent in the internal SERA unified framework;
	\item Conduct a study in a negotiation scenario to study the impact of cooperation.
\end{itemize}

In section X we introduce the background research to understand the concepts of rapport, social robotics and machine learning. In section Y, it will be described the state-of-art rapports agents developed by X, Y and Z. The proposed solution and how its quality will be evaluated will be described in section W. Lastly, the final conclusions and the main concerns will be address in section U.

The remainder of this paper is organized as follows. In
the next section, we discuss related work on learning social behavior synthesis models. We introduce the IPL approach in Section III. In Sections IV and V, we describe, respectively, the setup and the results of an experiment on the synthesis of backchannel timings. We conclude with Section VI.
