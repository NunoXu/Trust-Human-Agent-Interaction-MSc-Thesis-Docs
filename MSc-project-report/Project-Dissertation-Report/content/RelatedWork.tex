% More than a literature review
% Organize related work - impose structure
% Be clear as to how previous work being described relates to your own.
% The reader should not be left wondering why you've described something!!
% Critique the existing work - Where is it strong where is it weak? What are the unreasonable/undesirable assumptions?
% Identify opportunities for more research (i.e., your thesis) Are there unaddressed, or more important related topics?
% After reading this chapter, one should understand the motivation for and importance of your thesis
% You should clearly and precisely define all of the key concepts dealt with in the rest of the thesis, and teach the reader what s/he needs to know to understand the rest of the thesis.


\section{Related Work}
\label{sec:Related Work}
Computational Trust research has been focused on modelling trust in \acp{MAS}, specially on open e-commerce environments\cite{Granatyr2015, HanYu2013, Pinyol2013, Noorian2010, Huang2008}, with at least 106 models created\cite{Granatyr2015}, since the formalization of trust as a measurable property in 1994 \cite{Marsh1994}.




The work I will be comparing my work with (direct influence, biggest part of this document, has to make sense, it has to show the relevance of mine work in a international way)


In data-driven timing generation for social behaviors, using \ac{ML} techniques, it is retrieved two types of samples: positive samples representing the moments (or timings) in the interaction that are socially acceptable to generate a backchannel gesture (e.g., a head nod or a vocalization "hmm hmm") and negative samples as the moments when it is unacceptable. In previous approaches, corpus based, positive samples are taken directly from the annotated dataset and the negative randomly as long as they do not overlap with a positive sample. 



% O que fizeram?
% Como se enquadra?
% Porque é importante?
% Como pode ser útil?

\subsection{Discussion}
\label{subsec:RelWorkDiscussion}

Based on everything and what everyone is doing or done, these are the problems, these are the advantageous and disavantageous of the aproach.