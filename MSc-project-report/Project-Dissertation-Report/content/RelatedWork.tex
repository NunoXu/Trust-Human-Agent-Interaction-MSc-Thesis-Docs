% More than a literature review
% Organize related work - impose structure
% Be clear as to how previous work being described relates to your own.
% The reader should not be left wondering why you've described something!!
% Critique the existing work - Where is it strong where is it weak? What are the unreasonable/undesirable assumptions?
% Identify opportunities for more research (i.e., your thesis) Are there unaddressed, or more important related topics?
% After reading this chapter, one should understand the motivation for and importance of your thesis
% You should clearly and precisely define all of the key concepts dealt with in the rest of the thesis, and teach the reader what s/he needs to know to understand the rest of the thesis.


\section{Related Work}
\label{sec:Related Work}
Computational Trust research has been focused on modelling trust in \acp{MAS}, specially on open e-commerce environments\cite{Granatyr2015, HanYu2013, Pinyol2013, Noorian2010, Huang2008}, with at least 106 models created\cite{Granatyr2015}, since the formalization of trust as a measurable property by Marsh in 1994 \cite{Marsh1994}. We will present some models from which we will take inspiration while creating our own, 

\subsection{Trust Models}
\label{subsec:Related work:Trust Models}
For related work concerning Trust Models we will focus on \textbf{Cognitive} Trust Models, first introduced by Castelfranchi et. al.\cite{Castelfranchi1998}, as we want to focus on modelling trust through multiple dimensions, with the intent of having trust depend on the action to perform, context and agent performing the task and having these dimensions represented explicitly in the model, something that it is not possible with \textbf{Numerical}, like the one introduced by \cite{Marsh1994}.

\subsubsection{Castelfranchi et. al.}
Having introduced the concept of Cognitive Trust Models, this author's model is generally used as a basis for most other authors, % TODO Insert citation here, check ReGrET

\subsubsection{Repage}


\subsubsection{BDI + Repage}

\subsection{The Perception and Measurement of Human-Robot Trust}
\label{subsec:Related work:The Perception and Measurement of Human-Robot Trust}

Schaefer\cite{Schaefer2009} presents a trust perception scale, that provides a way of extracting an accurate trust score from humans interacting with robots. The scale is composed of 40 items that can be ranked from 0 to 100, in 10 point intervals. The final result it then averaged by adding all the item values and divided by the total number of items (40).



While this work has been done specifically for \ac{HRI} we believe that a sub-set of this items can be used for the features used in the cognitive model of the user's trust, further described in Section %todo give refence here





% O que fizeram?
% Como se enquadra?
% Porque é importante?
% Como pode ser útil?

\subsection{Discussion}
\label{subsec:RelWorkDiscussion}

Based on everything and what everyone is doing or done, these are the problems, these are the advantageous and disavantageous of the aproach.