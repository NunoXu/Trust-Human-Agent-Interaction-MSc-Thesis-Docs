\section{Proposed Solution}
\label{sec:Solution}
In this Section we will address the components to create and support a module capable of suggesting trustful actions to the agent, starting with Cognitive Trust Modelling Module, go on to the Trust Decision Making Module, and finally talk about how they connect together and to the rest of an agent architecture.


\subsection{Cognitive Trust Modelling Module}
\label{subsec:Solution:Trust Assessment Module}
In this module we aim to create a trust representation of the user. The model must be able to represent the user's beliefs while also provide an evaluation on how trustworthy is the agent. To do this, we will base the model based on the Repage architecture, described in Section \ref{subsubsec:Related work:Trust Models:Repage}, as it is the architecture, of very few, that we found closer to implementation, and because creating an architecture is not in the scope of this project, in conjunction to the severe time restraints.

This module will represent the memory and detector components of the architecture, and the concrete implementation for the beliefs is still under discussion, but our main candidate would be the BC-Logic described in \ref{subsubsec:Related work:Trust Models:BDI + Repage} by Pinyol \textit{et al.} \cite{Pinyol2009}.

\subsection{Trustful Action Suggestion Module}
\label{subsec:Solution:Trust Decision Making Module}
This module is the most roughly defined as it is the one that may still require some amount of experimentation to be properly designed, but the main goal is to suggest actions that will either improve the strength of existing beliefs on the trust model, or improve the trust value on the agent, occupying the analyser component in Repage.

\subsection{Putting it together}
Both modules must be integrated with the agent in which we will evaluate the work with. Depending on the current state of the agent, a translation module may have to be created, to connect the main memories of the agent to the memories in Cognitive Trust Modelling Module.