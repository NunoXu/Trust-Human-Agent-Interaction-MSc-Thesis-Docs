\section{Proposed Solution}
\label{sec:Solution}
In this Section we will address our current planning for our solution, and will start with Trust Assessment Module, in Section \ref{subsec:Solution:Trust Assessment Module}, go on to the Trust Decision Making Module, in Section and finally talk about how they connect together and to the rest of an agent architecture.



\begin{itemize}
	\item Architecture/Model: does not need to be very detailed but enough to understand what it is trying to be done. Tools being used, etc
	\item Evaluation: How the work will be evaluated.	
	\item Planning (GANTT maybe)	
\end{itemize}Aproach - Makes sense because you talked previously


As [paper do iterative perceptual learning], using hand crafted rules [7, 12] despite being intuitive use very shallow features and the development of these rules is not trivial.
E.g.

\subsection{Trust Assessment Module}
\label{subsec:Solution:Trust Assessment Module}
In this module we aim to create a trust representation of the user. The model must be able to represent the user's beliefs while also provide a ranking on how trustworthy is the agent. To do this, we will base the model on the work made by Pinyol\cite{Pinyol2009}, described in Section \ref{?}.



\subsection{Trust Decision Making Module}
\label{subsec:Solution:Trust Decision Making Module}

\subsection{Linking all together}
\label{subsec:Solution:Linking all together}