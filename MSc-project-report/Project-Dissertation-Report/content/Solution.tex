\section{Proposed Solution}
\label{sec:Solution}

\begin{itemize}
	\item Architecture/Model: does not need to be very detailed but enough to understand what it is trying to be done. Tools being used, etc
	\item Evaluation: How the work will be evaluated.	
	\item Planning (GANTT maybe)	
\end{itemize}Aproach - Makes sense because you talked previously


As [paper do iterative perceptual learning], using hand crafted rules [7, 12] despite being intuitive use very shallow features and the development of these rules is not trivial.
E.g.

\subsection{Subsection Heading Here}
Subsection text here.
A figure can be inserted like the example of Figure~\ref{fig_figure}.

According to previous studies [33 dont stare at me], during dyadic interactions, the listener usually maintains long gazes at the speaker and only interrupts briefly from time to time. 

In fact, [9 toward dyadic...] found that in a negotiation setting not reciprocating negative self-disclosure led to decreased feelings of rapport. 
\cite{Bronstein2012} 

mutual gaze in determine turn-taking turn-taking [8] [12] [14] [56] [47] [todos do dont stare at me]. 

One of the most notable non-verbal behaviors to build rapport is gaze because it is a clear signal of mutual attention (as our parents said, look to people eyes when talking to them), acts as an invitation to interaction, increases dynamism, likelability and believability [4 do dont stare at me]


%%%%%%%%%%%%%%%%%%%%%%%%%%%%%%%%%
%%%%%% RAW
%%%%%%%%%%%%%%%%%%%%%%%%%%%%%%%%%

%%5 dont stare at me %%%%%%
its impact in a wide range of interpersonal domains includ- ing social engagement [52], classroom learning [22], suc- cess in negotiations [20], improving worker compliance [18], psychotherapeutic effectiveness [59], and improved quality of child care [11].

Gaze as object of interest [8] [37] [55]. , effects on the way communication proceeds [54] [60] [23] [19] [28].

%\item Head gestures; %[7 do virtual rapport 2.0] refere q aumenta persuacao

%%%%%%%%%%%%%%
%%%%%%%%%%%%%%%%%%%%
%%%%%%%%%%%%%%


%%%%%%%%%
% POTENTIAL RAPPORT RULES

% dint stare at me
In fact, previous work demonstrated that there is evidence that in health domains, high rapport doctors engaged in less extensive eye-contact than low rapport doctors  , 85\% and 70\% of the interaction time respectively . However, the impact of the gaze depends if the interacts are in a helping context (e.g. meetings with a doctor) or in a non-helping context (e.g. interviewing)  [dont stare at me].  On the latter, directed gaze is correlated positively with participant’s evaluative impression [ Tickle-Degnan and Rosenthal]. In interviweing ocntexts, Goldberg, Kiesler, and Collins [25] found that people who spent more time gazing at an interviewer received higher socio- emotional evaluations. 

Argyle [1] found that in dyadic conversa- tions, the listener spent an average of about 75% of the time gazing at the speaker.

Kendon [33] reported that a typical pattern of interaction when two people converse with each other consists of the listener maintaining fairly long gazes at the speaker, interrupted only by short glances away.

In short, gaze can also have negative impact if not dosed correctly.
%%%%%%%%%%
%%%%%%%%%%%%
%%%%%%%%%%%%

%%%%%%%%%%%%
% EVALUATING
%%%%%%%%%%%%

To assess if negative arousal played some role, we asked the partici- pants to evaluate how uncomfortable they were when inte- racting with the agent (the embarrassment scale) in the post-questionnaire packet. ANOVA

%%%%%%%%%%%%