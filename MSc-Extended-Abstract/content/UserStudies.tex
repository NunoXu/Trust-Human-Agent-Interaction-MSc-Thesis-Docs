% This is "UserStudies.tex"
% A content sub-file for Trustfull Action Suggestion Thesis Extended Abstract

\section{User Studies}
\label{sec:UserStudies}

In order to evaluate the model we performed a user study in a Investor Game type scenario, with the intent to create an environment where the human participant would need to make a quantitative choice representing his trust in the agent. These studies were performed in collaboration with Henriques' for his work on \textit{Rapport Simulation in Agents} \cite{todo}, as the area of studies were similar enough to 

\subsection{Quick Numbers Scenario}
\label{subsec:QuickNumbersScenario}

\subsubsection{Agent's \ac{AI}}
An \ac{AI} was created to progress through the scenario, with it being mostly scripted reactionary events where the Agent would just press a button when perceiving some specific change in the scenario. But one part we should discuss is the \ac{AI} for the scenario's game, where although simple, some effort was given to make it parametrizable, in order to adjust the agent's ability in the game. The \ac{AI} was programmed to press a circle every $x$ seconds, with a $y$ chance of pressing the right circle. Throughout development and pilot testing, we empirically found 1 second to be a good value for $x$, as it made the agent have a human-like reaction time for pressing the circles, and 30\% a good percentage for $y$, because we wanted to make the agent's ability in the game also similar to a human.

\subsection{Trust Game}
\label{subsec:Trustgame}
Our overall scenario is very much based on the Trust/Investor Game, first proposed by Berg et al.~\cite{JoyceBergJohnDickhaut}, so we present a small introduction and discussion to the scenario. The game is set up with 2 anonymous players, which we will call player A and player B, where \$10 is given to player A and none to player B. In the first phase player A must choose how much of the starting \$10 should he give to player B knowing that the value will be tripled in player B’s hands. In the second phase player B chooses how much of the, now tripled, money will he return no player A.

This game forms a good base for our scenario because the decision of how much A should give to B is dependent on 2 different factors: the ability and willingness of B to multiply the investment and the willingness of B to return the profits of the investment. This is possible to perform because, while the source of the game explicit the multiplication factor of giving money to B, this value can be a hidden parameter, making the ability of B something to take into account.

We used this game as a foundation for our user studies, described further on in Section \ref{sec:UserStudies}, where we attempted to resolve some of it's faults, such as the game lack of any negotiation phase, in fact, both players are in separate rooms, with no way of interacting with one another, inducing trust to be modelled in a game theory point of view, which is contrary what we wish to accomplish.

\subsection{Results}
% Mostly inconclusive