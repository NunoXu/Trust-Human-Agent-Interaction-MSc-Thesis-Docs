% This is "Model.tex"
% A content sub-file for Trustfull Action Suggestion Thesis Extended Abstract

\section{Trust Model}
\label{sec:TrustModel}
We sought out to develop a trust model definition that would be easily implementable, but generic enough to be able to adapt to various testing scenarios. To do this we first took inspiration from the work by Sabater et al.\cite{Sabater2006} described in Section \ref{TODO} by taking a similar approach to architecture. The model can be divided in 3 main components:
\begin{itemize}
    \item Core Model, 
    \item Perceptions 
    \item Suggestions,
\end{itemize}

\subsection{Architecture}


% Agent contains Trustees, which are representations of the other agents.
% A trustee has a set of features that the trustor has assigned as representative of it's trust.
% A feature belongs to a certain category, which in most scenarios, would be Ability and Willingness
% A feature has a belief value, that is calculated from a set of belief sources
% A belief source can either be Direct Contact, Reputation or Bias
% Each belief source has three values, a belief value, a certainty value, and a time value.
% Belief value is a normalized number that describes the trustee's evaluation on that feature
% Certainty describes how well the trustee was evaluated, in Reputation for instance, this might represent how well we trust in the reporter, and in Direct Contact how well the trustor observed the trustee performing said feature.
% Time is just a record of when was this belief source recorded, as older records might have a lower impact in the overall belief value score, compared to newer records.
% Perceptions correspond to the inputs from the environment that are inserted into the model as belief values to associated features.
% 

\subsection{Representing Trust Features}

\subsection{Action Suggestion}
% This model component describes a mapping of trust increasing oppurtunities to actions that are performed depending on what features are lacking in the trustee for a particular trustor.

% A perception is mapped to relevant actions, and actions are allocated depending on time available and what is the lowest scoring relevant feature.

