% #############################################################################
% RESUMO em Português
% !TEX root = ../main.tex
% #############################################################################
% use \noindent in firts paragraph
\noindent Confiança é um ingrediente essencial para cooperação e colaboração, portanto se quisermos continuar o desenvolvimento de agentes autónomos colaborativos, necessitamos de abordar a questão da confiança nestas relações. Por esta razão, confiança computacional em \ac{IHR} tem visto um aumento súbito de interesse nos últimos anos, contudo a literatura tem se apenas concentrado em assuntos como desenho, animação e modelagem. Esta tese aborda o assunto inexplorado de ativamente melhorar confiança através da sugestão de ações confiáveis ao agente virtual. De encontro a este objetivo, nós desenvolvemos um Modelo de Confiança Cognitivo capaz de representar um modelo de Confiança baseado em crenças e de sugerir falas com o objetivo de melhorar confiança. Para além disso, nós também desenhamos um cenário original avaliador de Confiança e Rapport, o Quick Numbers, para conseguirmos de forma adequada avaliar os efeitos da vontade em Confiança e ter uma medida simples de Trust embebida no cenário de testes. Finalmente, descrevemos o estudo com utilizadores efetuado para avaliar o Modelo de Confiança usando o cenário Quick Numbers.