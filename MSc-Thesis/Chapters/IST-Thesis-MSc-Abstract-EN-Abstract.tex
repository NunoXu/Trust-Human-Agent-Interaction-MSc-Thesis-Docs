% #############################################################################
% Abstract Text
% !TEX root = ../main.tex
% #############################################################################
% use \noindent in firts paragraph
\noindent  Trust is an essential ingredient for cooperation and collaboration, so if we want to further develop autonomous collaborative agents, we must address the issue of trust in such relationships. For that reason, computational trust in \ac{HRI} has seen a great spike of interest in recent years, however the literature has been only focused in issues like design, animation and modelling. This thesis addresses the uncharted matter of actively improving trust by suggesting trustful actions to the agent. Towards this goal we developed a Cognitive Trust Model capable of representing a belief based Trust model and suggest utterances with the goal of improving Trust. Furthermore we also designed a novel Trust and Rapport evaluating scenario, Quick Numbers, to be able to properly evaluate the effect of willingness in Trust and have a simple measurement of Trust embedded in the testing scenario. Finally, we describe the User Studies to evaluate the Trust Model using the Quick Numbers scenario.