% #############################################################################
% This is Chapter 7
% !TEX root = ../main.tex
% #############################################################################
% Change the Name of the Chapter i the following line
\fancychapter{Conclusions}
\label{chap:Conclusions}
Throughout this thesis we addressed the work done to develop a Cognitive Trust Model capable of suggesting actions to improve Trust in a virtual agent. We first went through our thoughts about the lack of research done in the area of trust in \ac{HRI}, especially regarding trust improvement, and what we propose to address that issue. Then we went on to establish some background in \ac{HRI} concepts specific to the domain, like the various definitions of Trust, while going through some discussion as what was more appropriate for our work. In the next chapter we delved into some of the Cognitive Trust Models that we found. While there were many more aside from those discussed in this thesis, we wanted to focus on the ones that most closely related with the one we developed. Following that we presented our Trust Model, describing its 3 main components: Memory, Perception and Action Suggestion, and how they interact to compose our model. Due to problems finding a suitable evaluation scenario for our model, we then describe our second contribution of this thesis, a novel Trust and Rapport evaluation scenario, Quick Numbers, that aims to evaluate how ability and willingness jointly affect trust, by imposing to the participant the choice to entrust the agent, with their own earned resources, to play a game, in which the participant has some idea of the agent's ability. Finally we showed how we used the scenario to perform User Studies on the Trust Model. Unfortunately, the results can be considered either inconclusive, or right against our effort to improve Trust, as there was no apparent change in Trust measurements in conditions with and without the Action Suggestion module. Nevertheless, we believe that our contributions were significant by providing an implementable base from which other research projects in the same area may start. 


\section{Future Work}
The first step to take would be to create a complete implementation of the model, with Actions suggesting social strategies instead of just utterances. 

Afterwards, the model should be tested in larger variety of scenarios, specially in studies with returning participants, to test and improve the consequences of time passage when calculating Trust. The transferability of trust between scenarios should also be tested, by testing between scenarios with task related to similar features.

Another worthwhile addition to the model would be the design and implementation of Hierarchical Trust, as Trust Features that encompass broader terms, would be connected to hyponym Trust Features and take them into account when calculating Trust.